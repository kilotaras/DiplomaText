\documentclass[a4paper,oneside]{article}

%%%%%%%% TEXT packages
\usepackage{cmap}
\usepackage[T1,T2A]{fontenc}
\usepackage[utf8]{inputenc}
\usepackage[ukrainian]{babel}


%% Packages for Graphics & Figures %%%%%%%%%%%%%%%%%%%%%%%%%%
\usepackage[pdftex]{graphicx} %%For loading graphic files

\usepackage[unicode]{hyperref}
\hypersetup{
    colorlinks,
    citecolor=black,
    filecolor=black,
    linkcolor=black,
    urlcolor=black
}

%% Math Packages %%%%%%%%%%%%%%%%%%%%%%%%%%%%%%%%%%%%%%%%%%%%
\usepackage{amsmath}
\usepackage{amsthm}
\usepackage{amsfonts}

\usepackage{relsize}

%!TEX root = Main.tex

%page geometry and font size
\usepackage[14pt]{extsizes}
\usepackage[top=2cm,left=3cm,right=1.5cm,bottom=2cm]{geometry}

%indents
\usepackage{indentfirst}
\setlength{\parindent}{1cm}

%section style
\usepackage{titlesec}
\titleformat{\section}{\Large \bf}{\thesection .}{0.5em}{\centering}[]
\titleformat{\subsection}{\large \bf}{\hspace{30pt}\thesubsection}{0.5em}{}[]
\titleformat{\subsubsection}{\large}{\hspace{30pt}\thesubsubsection}{0.5em}{}[]

\newcommand{\sectionbreak}{\clearpage}

%no additional numbers in TOC
\setcounter{secnumdepth}{3}

%insert figures at place
\usepackage{float}


%bibliography in TOC
\usepackage{tocbibind}
\renewcommand{\tocbibname}{Список використаної літератури}

\usepackage{fancyhdr}
\fancyhf{}
\renewcommand{\headrulewidth}{0pt}
\makeatletter
\let\ps@plain\ps@fancy
\makeatother
\fancyhead[R]{\thepage}




%names for references

\def\figureautorefname{Рис.}%



\regtotcounter{figure}
\regtotcounter{table}

\newtotcounter{citenum}
\def\oldcite{}
\let\oldcite=\bibcite
\def\bibcite{\stepcounter{citenum}\oldcite}

%!TEX root = Main.tex


\newcommand{\norm}[1]{\left\lVert#1\right\rVert}

\newcommand{\R}{\mathbb{R}}

% 1.1, 1.2, 2.1, ...
\numberwithin{equation}{section}


\newcommand\sect[1]{
  \clearpage
  \phantomsection
  \addcontentsline{toc}{section}{#1}
  \section*{#1}%
}

\newcommand\subsect[1]{
  \phantomsection
  \addcontentsline{toc}{subsection}{#1}
  \subsection*{#1}%
}

\newcommand{\undef}[1]{\let#1\undefined}

\newcommand{\TODO}[1]{\colorbox{red}{TODO: #1}}

\newcommand{\nn}[1]{\pgfmathprintnumber{#1}}





\begin{document}
	\pagestyle{empty} %No headings for the first pages.

	%!TEX root = ../Main.tex
\includepdfmerge{include/TitleImport.pdf}


	\linespread{1.5}
	\selectfont
	\addtocounter{page}{1}

	\section*{Зміст}

	\makeatletter
	\@starttoc{toc}
	\makeatother

	\newpage
	\pagestyle{plain}

	%!TEX root = ../Main.tex

\begin{frame}{Вступ}
	\begin{itemize}
		\item МСЕ - числова техніка знаходження розв'язків РЧП
		\item АОП - важлива частина сучасної інформатики
			\begin{itemize}
				\item{ідея виникла ще в 1978 \cite{babuvska1978posteriori}}
				\item активно розвиваються останнім часом \cite{ainsworth2011posteriori, eriksson1995introduction,verfurth1994posteriori,verfurth1996review}
			\end{itemize}
		\item в роботі будуємо
			\begin{itemize}
				\item АОП для задачі дифузії-адвекції-реакції
				\item адаптивну схему для МСЕ
			\end{itemize}
	\end{itemize}
\end{frame}

	%!TEX root = ../Main.tex
\clearpage
\section{Формулювання задачі}

Нехай $\Omega$ - область в Евклідовому просторі $\R^d$ з неперервним краєм $\Gamma \equiv \delta \Omega$, яка задовольняє умову Ліпшица.

Розглянемо крайову задачу \eqref{eq:general_boundary} та її варіаційне формулювання \eqref{eq:general_variational}:
%
\begin{equation}\label{eq:general_boundary}
	\begin{cases}
			- \nabla (\mu \nabla u) + \beta \cdot \nabla u + &\sigma u = f,  \\
			&u|_\Gamma = 0.
	\end{cases}
\end{equation}
%
\begin{equation}\label{eq:general_variational}
	\begin{cases}
		\mbox{для } V := H_0^1 \left( \Omega \right) =
		\left\lbrace
			v \in H^1 \Omega : v|_\Gamma = 0
		\right\rbrace, \\

		\quad a_\Omega(w,v) := \displaystyle\int_\Omega
		\left[
			\mu \nabla w \nabla v +v \beta \nabla w + \sigma wv
		\right] dx, \\

		\qquad \left\langle l_\Omega, v \right\rangle := \displaystyle\int_\Omega fvdx \quad \forall v,w \in V \\

		\mbox{знайти }u \in V \mbox{, таку що} \\

		\qquad a_\Omega(w,v) = \left\langle l_\Omega, v \right\rangle \quad \forall v \in V.

	\end{cases}
\end{equation}

Тут $\nabla u := \left\lbrace \frac{\partial u}{\partial x_i} \right\rbrace_{i=1}^d$,
	а також $\mu = \left\lbrace \mu_{ij}(x) \right\rbrace_{i,j=1}^d$,
	$\beta = \left\lbrace \beta_i(x) \right\rbrace_{i=1}^d$,
	$\sigma = \sigma(x)$,
	$f = f(x)$ - задані функції, такі що
%
%
\begin{equation}
	\begin{cases}
		\mu_{ij}(x) = \mu_{ji}(x) ,\\
			\sum \limits_{i,j=1}^d \mu_{ij}(x) \xi_i \xi_j
				\ge
			\mu_0 \sum \limits_{i=1}^d \xi_i^2, \quad
			\mu_0 =\text{const}, \quad \forall \xi_i \in \R ,\\
		\nabla \cdot \beta(x) := \mathlarger{\sum \limits_{i=1}^d} \frac{\partial \beta_i(x)}{\partial x_i} = 0
			\text{ майже скрізь в } \Omega.
	\end{cases}
\end{equation}
%
\begin{equation}\label{eq:coef_spaces}
	\mu_ij, \beta_i, \sigma \in L^\infty(\Omega), \quad f \in L^2(\Omega)
\end{equation}
%
При таких умовах білінійна форма $a_\Omega (\cdot,\cdot): V \times V  \to \R$ є неперервною від вхідних даних, $V$-еліптичною і генерує енергетичну норму \cite{kozarevska2002}
%
\begin{equation}\label{eq:energy_norm}
	\norm{u}_V := \sqrt{a_\Omega(u,u)}
\end{equation}

\newcommand{\vonenorm}{\left|v\right|_{1,\Omega}}

Норма \eqref{eq:energy_norm} є еквівалентною до норми $\vonenorm = \sqrt{(\nabla v, \nabla v)}$ :
%
\newcommand{\infnorm}[1]{\norm{#1}_\infty}
%
\begin{equation}\label{eq:norm_equivalence}
	\mu_0 \vonenorm^2 \le \norm{u}_V \le
		\mu_0
			\left[
				\frac{\norm{\mu_0}_\infty}{\mu_0}+Pe(1+Sh)
			\right]
		\vonenorm^2 ,
\end{equation}
%
де
%
\begin{equation}
	Pe:= \frac{\infnorm{\beta} \text{ diam} \Omega}{\mu_0}, \qquad
	Sh:= \frac{\infnorm{\sigma} \text{ diam} \Omega}{\infnorm{\beta}}
\end{equation}
%
критерії Пекле і Струхаля.
%
\begin{equation}
	\begin{split}
		\infnorm{\sigma} := ess \sup \limits_{x \in \Omega} |\sigma(x)|; \\
		\infnorm{\beta} := \sqrt{\sum \limits_{i=1}^d \infnorm{\beta_i}^2}; \\
		\infnorm{\mu} := \sqrt{\sum \limits_{i,j=1}^d \infnorm{\mu_{i,j}}^2}.
	\end{split}
\end{equation}

\undef{\vonenorm}

Згідно з
\eqref{eq:coef_spaces} лінійний функціонал $l : V \to \R$ є неперервним.
Тому згідно теореми Лакса-Мільграма-Вишика задача
\eqref{eq:general_variational} коректно поставлена. Вона має єдиний розв'язок $u \in V$, що неперервно залежить від вхідних даних.

	%!TEX root = ../Main.tex

\begin{frame}[allowframebreaks]
	\frametitle<presentation>{Дискретизована задача}
		 $\Omega \subset \R$ поділена на скінченні елементи $K$:
		 $
			T_h=\lbrace K \rbrace, \quad
			h := \max_{K \in T_h}h_K, \quad
			h_k := \mbox{diam} K
		$
		\begin{equation}\label{eq:split_properties}
			\begin{split}
				& \Omega = \bigcup_{K \in T_h} K, \\
				& K \bigcap K^\prime = \emptyset \quad \forall K, K^\prime \in T_h : K \neq K^\prime, \\
				& \overline K \bigcap \overline {K^\prime} =
				\begin{cases}
					S := \mbox{\{спільне ребро\}}, \\
					A := \mbox{\{спільна вершина\}},  \\
					\emptyset. \\
				\end{cases}
			\end{split}
		\end{equation}

	\framebreak

		$V_h \subset V$, $\mbox{dim} V_h = N(h) = N < + \infty$ - простір апроксимацій

		Заміняємо задачу \eqref{eq:general_variational} на наступну

		\begin{equation}\label{eq:general_discrete}
			\begin{cases}
				\mbox{для заданого } T_h = \{K\}, V_h  \subset V \\
				\mbox{знайти } u_h \in V_h \mbox{, таку що} \\
				a_\Omega(u_h, v) = \langle l_\Omega, v \rangle, \qquad \forall v \in V_h.
			\end{cases}
		\end{equation}

	\framebreak

		Обравши базис $\lbrace \phi_i(x)\rbrace_{i=1}^N \subset V_h$,конкретизуємо \eqref{eq:general_discrete}:
		%
		\begin{equation}\label{eq:concrete_discrete}
		\begin{cases}
			\text{для заданих } T_h = \lbrace K \rbrace \text{ і базису } \lbrace \phi_i(x)\rbrace_{i=1}^N \text{ з простору } V_h \subset V \\
			\text{знайти коефіцієнти } q_m \in \R \text{ апроксимації } \\
				\qquad u_h(x) := \sum\limits_{m=1}^N q_m \phi_m (x) \in V_h, \text{ такі що} \\
			\sum\limits_{m=1}^N a_\Omega(\phi_m, \phi_i)q_m = \langle l_\Omega, \phi_i \rangle, \qquad i = 1,\dots,N.
		\end{cases}
		\end{equation}

		Cистема рівнянь буде додатньо визначеною \cite{OstShynAee11}

\end{frame}

	\section{Задача оцінки похибки}
	\section{Апостеріорний оцінювач}
	\section{Апостеріорний оцінювач на трикутнику}
		\subsection{Лінійна МСЕ на трикутнику}
		\subsection{Оцінювач на трикутнику}
		\subsection{Знаходження значень оцінювач на трикутнику}
		\subsection{Знаходження значень оцінювач на трикутнику}
	\section{Числові результати}
		\subsection{Задача 1}
		\subsection{Задача 2}
	\sect{Висновки}
	\sect{Список літератури}
\end{document}
