\documentclass[12pt]{beamer}

%%%%%%%% TEXT packages
\usepackage{cmap}
\usepackage[utf8]{inputenc}
\usepackage[T1,T2A]{fontenc}
\usepackage[ukrainian]{babel}


\hypersetup{
	unicode=true
}

%% Math Packages %%%%%%%%%%%%%%%%%%%%%%%%%%%%%%%%%%%%%%%%%%%%
\usepackage{amsmath}
\usepackage{amsthm}
\usepackage{amsfonts}

\usepackage{relsize}
\usepackage{import}

\usepackage{textcomp}

\usepackage{pgfplotstable}
\pgfplotsset{compat=1.10}
\usepackage{booktabs}
\usepackage{longtable}

\usepackage{textcase}

\usepackage{pdfpages}
\usepackage{lastpage}
\usepackage{totcount}

%!TEX root = Main.tex


% %insert figures at place
\usepackage{float}
%\usepackage{floatrow}
\def\figureautorefname{Рис.}%

\pgfkeys{pgf/number format/.cd, set thousands separator={\,}}

\setbeamertemplate{caption}[numbered]

\setbeamertemplate{section page}
{
    \begin{centering}
    \begin{beamercolorbox}[sep=12pt,center]{part title}
    \usebeamerfont{section title}\insertsection\par
    \end{beamercolorbox}
    \end{centering}
}

\setbeamertemplate{subsection page}
{
    \begin{centering}
	    \begin{beamercolorbox}[sep=12pt,center]{part title}
	    	\usebeamerfont{section title}\insertsection\par\par
	    	\small{\usebeamerfont{subsection title}\insertsubsection\par}
	    \end{beamercolorbox}
    \end{centering}
}

%!TEX root = Main.tex


\newcommand{\norm}[1]{\left\lVert#1\right\rVert}

\newcommand{\R}{\mathbb{R}}

% 1.1, 1.2, 2.1, ...
\numberwithin{equation}{section}


\usetheme{Warsaw}
\usecolortheme{crane}
\setbeamertemplate{bibliography item}[text]

\title[Апостеріорні оцінювачі похибок МСЕ]{Апостеріорні оцінювачі похибок апроксимацій МСЕ та їх застосування}
\author{Тарас Бойко}
\date{16 червня 2014}

\begin{document}


	%!TEX root = ../Main.tex

\begin{frame}
\titlepage
\end{frame}


	%!TEX root = ../Main.tex

\sect{Вступ}

Апостеріорні оцінювачі для методу скінченних елементів(МСЕ) є важливою частиною сучасної інформатики (див. \cite{verfurth1996review, verfurth1994posteriori,eriksson1995introduction,ainsworth2011posteriori}). За останні десятиліття на базі початкової ідеї Бабушки і Рейнбольдта \cite{babuvska1978posteriori} було побудовано цілу сім'ю різних апостеріорних оцінювачів похибки (АОП). З їх допомогою можна якісно оцінити похибку отриманої апроксимації МСЕ і створити базу для локального покращення сітки, що дає змогу знаходити розв'язки з заданою точністю і мінімальними обчислювальними затратами \cite{babuska2011finite}

В даній роботі ми будуємо апостеріорний оцінювач похибки для двовимірної задач та адаптивну схему для МСЕ на його основі. Для побудованої схеми проводимо числовий аналіз якості для задачі з відомим точним розв'язком.
Отримані результати порівнюються з результатами роботи оцінювача запропонованого в \cite{OstShynAee11}



	%!TEX root = ../Main.tex

\newcommand{\vonenorm}{\left|v\right|_{1,\Omega}}
\newcommand{\infnorm}[1]{\norm{#1}_\infty}

\section{Формулювання крайової і варіаційної задач}
\frame{\sectionpage}
\begin{frame}
	\frametitle<presentation>{Крайова і варіаційна задачі}

		 \begin{equation}
			\begin{cases}
				- \nabla (\mu \cdot \nabla u) + \beta \cdot \nabla u + &\sigma u = f, \text{ в } \Omega \subset \R^d \\
				&u = 0, \text{ на } \Gamma \equiv \delta \Omega
			\end{cases}
		\end{equation}

		\begin{equation}\label{eq:general_variational}
			\begin{cases}
				\mbox{Для } V := H_0^1 \left( \Omega \right) =
				\left\lbrace
					v \in H^1 \Omega : v|_\Gamma = 0
				\right\rbrace, \\

				\quad a_\Omega(w,v) := \displaystyle\int_\Omega
				\left[
					\mu \nabla w \cdot \nabla v +v \beta \cdot \nabla w + \sigma wv
				\right] dx, \\

				\qquad \left\langle l_\Omega, v \right\rangle := \displaystyle\int_\Omega fvdx \quad \forall v,w \in V \\

				\mbox{знайти }u \in V \mbox{, таку що} \\

				\qquad a_\Omega(w,v) = \left\langle l_\Omega, v \right\rangle \quad \forall v \in V.
			\end{cases}
		\end{equation}

\end{frame}

\begin{frame}{Регулярність даних задачі}
		%
		\begin{align}
			&\mu = \left\lbrace \mu_{ij}(x) \right\rbrace_{i,j=1}^d \qquad &\beta = \left\lbrace \beta_i(x) \right\rbrace_{i=1}^d \\
			&\sigma = \sigma(x) \qquad \qquad \qquad &f = f(x)
		\end{align}

		%
		%
		\begin{equation}
			\begin{cases}
				\mu_{ij}(x) = \mu_{ji}(x) ,\\
					\sum \limits_{i,j=1}^d \mu_{ij}(x) \xi_i \xi_j
						\ge
					\mu_0 \sum \limits_{i=1}^d \xi_i^2, \quad
					\mu_0 =\text{const} > 0, \quad \forall \xi_i \in \R ,\\
				\nabla \cdot \beta(x) := \mathlarger{\sum \limits_{i=1}^d} \frac{\partial \beta_i(x)}{\partial x_i} = 0
					\text{ майже скрізь в } \Omega.
			\end{cases}
		\end{equation}
		%
		\begin{equation}\label{eq:coef_spaces}
			\mu_{ij}, \beta_i, \sigma \in L^\infty(\Omega), \quad f \in L^2(\Omega)
		\end{equation}

\end{frame}

\begin{frame}{Властивості задачі I}

		$a_\Omega (\cdot,\cdot): V \times V  \to \R$ володіє наступними властивостями \footfullcite{kozarevska2002}:

		\begin{itemize}
			\item неперервна в $V \times V$
			\item є $V$-еліптичною
			\item генерує енергетичну норму
				\begin{equation}\label{eq:energy_norm}
					\norm{u}_V := \sqrt{a_\Omega(u,u)}
				\end{equation}
		\end{itemize}
\end{frame}


\begin{frame}{Властивості задачі II}
	Енергетична норма \eqref{eq:energy_norm} еквівалентна $\vonenorm = \sqrt{(\nabla v, \nabla v)}$ \footfullcite{OstShynAee11}
% %
 	\begin{equation}\label{eq:norm_equivalence}
		\mu_0 \vonenorm^2 \le \norm{v}^2_V \le
			\mu_0
				\left[
					\frac{\norm{\mu}_\infty}{\mu_0}+Pe(1+Sh)
				\right]
			\vonenorm^2 ,
	\end{equation}

	\begin{equation}
		Pe:= \frac{\infnorm{\beta} \text{ diam} \Omega}{\mu_0}, \qquad
		Sh:= \frac{\infnorm{\sigma} \text{ diam} \Omega}{\infnorm{\beta}}
	\end{equation}

	\begin{itemize}
		\item $l : V \to \R$ - неперервний функціонал
		\item згідно теореми Лакса-Мільграма-Вишика варіаційна задача \eqref{eq:general_variational} коректно поставлена
	\end{itemize}

\end{frame}

\undeff{\vonenorm}

	 %!TEX root = ../Main.tex

\section {Дискретизована задача}

Допустимо, що $\Omega \subset \R$ поділена в скінченні елементи $K$ таким чином, що поділ
$
	T_h=\lbrace K \rbrace, \quad
	h := \max_{K \in T_h}h_K, \quad
	h_k := \mbox{diam} K
$
має наступні властивості:

\begin{equation}
	\Omega =
\end{equation}



	% %!TEX root = ../Main.tex

\section{Задача про похибку апроксимацій МСЕ}

Враховуючи
\eqref{eq:general_variational} та дискретизовану версію
\eqref{eq:general_discrete} ми отримуємо наступну задачу:
%
\begin{equation}\label{eq:AE_problem}
	\begin{cases}
		\mbox{для заданого } T_h=\{K\} \text{, і відповідного наближення } u_h \in V_h \\
		\text{знайти похибку } e:=u-u_h \in E := V \setminus V_h \text{, таку що} \\
		a_\Omega(e,v) = \langle\rho(u_h), v\rangle \qquad \qquad \forall v \in E \text{, де}\\
		\langle\rho(w), v\rangle := \langle l_\Omega, v\rangle - a_\Omega(w, v) \quad \forall w,v \in V.
	\end{cases}
\end{equation}

Залишок $\langle\rho(u_h), v\rangle$ можна записати наступним чином
\begin{equation*}
	\langle\rho(u_h), v\rangle
		= \langle l_\Omega, v\rangle - a_\Omega(u_h, v)
		= a_\Omega(u,v)-a_\Omega(u_h, v) = a_\Omega(u-u_h, v)
\end{equation*}

Можна показати \cite{kvasnyca2002}, що
%
\begin{equation}\label{eq:residual_properties}
	\begin{split}
		\langle\rho(u_h), v\rangle = a_\Omega(u-u_h, v) = 0 \quad \forall v \in V_h \subset V \\
		\langle\rho(u_h), u-u_h\rangle = a_\Omega(u-u_h, u-u_h) = \norm{u-u_h}^2_V \\
		\norm{\rho(u_h)}_* = \sup \limits_{v \ne 0 \in V} \frac{|\langle\rho(u_h), v\rangle|}{\norm{v}_V} =
			\sup \limits_{v \ne 0 \in V} \frac{a_\Omega(u-u_h, v)}{\norm{v}_V} \le \norm{u-u_h}_V
	\end{split}
\end{equation}

З
\eqref{eq:residual_properties} слідує, що норма залишку еквівалента енергетичній нормі МСЕ
\begin{equation}
	\norm{\rho(u_h)}_* = \norm{u-u_h}_V
\end{equation}



	% %!TEX root = ../Main.tex

\section{Апостеріорна оцінка похибки}

Для розв'язку задачі
\eqref{eq:AE_problem} використаємо процес Гальоркіна вибравши певний скінченновимірний простір $E_h \subset E$

\begin{equation}\label{eq:AEE_formulation}
	\begin{cases}
		\mbox{для заданого } T_h=\{K\} \text{, і відповідного наближення } u_h \in V_h \\
		\text{та підпростору } E_h \subset E:=V \setminus V_h, \quad \text{dim} E_h = M(h) < +\infty \\
		\text{знайти оцінку } e_h \in E_h \text{, що} \\
		a_\Omega(e,v) = \langle\rho(u_h), v\rangle \qquad \forall v \in E_h
	\end{cases}
\end{equation}

Задача
\eqref{eq:AEE_formulation} коректно поставлена і її розв'язок $e_h \in E_h$ володіє наступними властивостями:

\begin{equation}
	\begin{split}
		\norm{e_h}_V^2 = a_\Omega(e_h, e_h) = \langle\rho(u_h), e_h\rangle \le \\
		\le \norm{\rho(u_h)}_* \norm{e_h}_V \le \norm{u-u_h}_v \norm{e_h}_V
	\end{split}
\end{equation}

тобто

\begin{equation}
	\norm{e_h}_V \le \norm{u-u_h}_V \quad \forall h > 0
\end{equation}

Обираючи базис $\lbrace \phi_i(x)\rbrace_{i=1}^M \subset E_h$ ми конкретизуємо задачу
\eqref{eq:AE_problem} до алгебраїчного представлення

\begin{equation}
	\begin{cases}
		\text{для заданих } T_h = \lbrace K \rbrace \text{ і базису } \lbrace \phi_i(x)\rbrace_{i=1}^N \text{ з простору } E_h \subset E\\
		\text{знайти коефіцієнти } \lambda_m \in \R \text{ лінійної комбінації } \\
			\qquad \qquad \qquad e_h(x) := \sum\limits_{m=1}^M \lambda_m \phi_m (x) \in E_h, \text{ що} \\
		\sum\limits_{m=1}^M a_\Omega(\phi_m, \phi_i)\lambda_m = \langle \rho(u_h), \phi_i \rangle \qquad i = 1..M(h)
	\end{cases}
\end{equation}

Очевидно, що якість отриманого оцінювача $e_h \in E_h$ напряму залежить від повноти простору $E_h \subset E$ та ефективності обрахунків.
Ефективність досягається завдяки схемі скінченних елементів, що дає змогу вибрати базис близький до ортогонального.

Далі ми наведемо приклад чисельної схеми, що задовільняє ці критерії у двовимірному випадку ($d=2$).

	% \import{include/}{6AeeOnTriangle}
	% %!TEX root = ../Main.tex

\section{Стратегія адаптування тріангуляції}
\frame{\sectionpage}

\begin{frame}{Стратегія адаптації}

	\begin{itemize}
		\item Отримавши оцінку похибки будуємо адаптивну стратегію для знаходження розв'язку із заданою точністю
		\item Починаємо з якоїсь сітки
		\item Ущільнюємо сітку на елементах де велика похибка
	\end{itemize}

\end{frame}

\begin{frame}{Умова поділу}

	Маємо
	%
	\begin{equation}\label{eq:error_element}
		e_h(x)\bigg|_K = \error{ij}+\error{jk}+\error{ki}, \qquad K = A_iA_jA_k.
	\end{equation}

	Показник якості розв'язку на трикутнику:
	%
	\begin{equation}\label{eq:error_indicator}
		\text{ind}_K = \sqrt{N \frac
				{\norm{e_h}^2_{H(K)}}
				{\norm{u_h+e_h}^2_H(\Omega)}
		}*100\%.
	\end{equation}

	\begin{equation*}\label{eq:error_indicator}
		\text{ind}_K < \varepsilon \Rightarrow \text{ ділимо трикутник}
	\end{equation*}

\end{frame}

\undeff{\error}

	% \import{include/}{8NumResults}

	% %!TEX root = ../Main.tex

\section{ОХОРОНА ПРАЦІ ПІД ЧАС НАПИСАННЯ МАГІСТЕРСЬКОЇ РОБОТИ}

\subsection{Вступ}

Специфіка написання кваліфікаційної роботи передбачає безліч факторів небезпеки, що зумовлені як безпосереднім процесом розробки проблеми та її вирішення через наукову концептуалізацію, так і через виконання функціональних обов’язків, які зумовлюються цим процесом.

Під час розгляду стану умов праці необхідно зупинитись на кількох важливих моментах. Оскільки робота виконувалась на комп’ютері, то і умови праці були облаштовані у відповідності до вимог щодо такого роду праці. Робота з комп'ютером вимагає значної розумової напруги і супроводжується нервово-емоційним навантаженням.

Більшість з цих проблем можуть бути зведені до мінімуму або усунуті взагалі у разі правильної організації робочого місця, створенні задовільних санітарно-гігієнічних умов праці, дотримання правил техніки безпеки і раціонального розподілу робочого часу.

\subsection{Аналіз стану умов праці}

\subsubsection{Характеристика виробничого середовища та його чинників}

Робоче  місце знаходиться у житловій будівлі, на другому поверсі. Загальна площа приміщення складає 12 $\text{м}^2$.

У кімнаті є два місця для праці, одне з яких обладнане персональним комп’ютером.
Параметри мікроклімату: середня температура повітря у літній період: 24 градуси за Цельсієм, зимою – 21 градус. Швидкість руху повітря не перевищує
 0,2 м/с. Розміщення вікон забезпечує природне освітлення з коефіцієнтом природного освітлення не менше 1,5\%, а загальне штучне освітлення, яке здійснюється за допомогою чотирьох люмінесцентних ламп і двох настільних лампи.

Характеристика систем вентилювання – змішана. Опалення – автономне.

У кімнаті є електрична мережа з напругою 220 В, яка створює небезпеку ураження електричним струмом. ПК та периферійні пристрої можуть бути джерелами електромагнітних  випромінювань, аерозолів та шкідливих речовин.

У кімнаті проводиться вологе прибирання два рази на тиждень.

В будинку присутній газовий котел Junior-Euroline. В квартирі наявне підключення до водоканалізаційної, електричної та газової мереж. Однак не передбачено наявність первинних засобів пожежогасіння.


\subsubsection{Опис процесу праці}

Процес праці в такому контексті передбачає наступні моменти:

1. Фізичне навантаження. В даному положенні варто зазначити, що переважна частина роботи виконується у сидячому стані і основне навантаження спрямовується на спину.

2. Нервово-психічна напруженість праці. Основна напруга спрямовується на зір, а також зумовлюється великим обсягом та складністю оброблюваної інформації. Водночас варто зазначити і те, що тривалість такої зосередженої роботи теж велика і її складно підрахувати, адже на протязі затвердженого терміну для її виконання часто поєднується із семестровим навантаженням на виконавця. Така ж неоднозначність стосується і тривалості технічних перерв протягом робочого дня – такі перерви носять нерегламентований характер. Щодо шумового забруднення, то останнє часто зумовлюється і потребою урізноманітнити робочий процес.

Написання кваліфікаційної роботи належить до легкої категорії робіт за ступенем важкості праці. Вона виконувалась сидячи. Щодо характеру організування виконання дипломної роботи, то він підпадає під нав’язаний режим, оскільки певні розділи роботи необхідно виконати у встановлені конкретні терміни. За ступенем нервово-психічної напруги виконання роботи можна віднести до другого–третього ступеня і кваліфікувати як помірно напружений–напружений за умови успішного виконання поставлених завдань.

\subsubsection{Аналіз методів дослідження, обладнання та характеристика речовин}

Під час написання роботи з використовувався ПК.

Роботу ПК та периферійних пристроїв супроводжує виділення багатьох хімічних речовин, зокрема озону, оксидів  нітрогену та аерозолів. В умовах роботи з ПК виникають наступні небезпечні та шкідливі чинники: несприятливі мікрокліматичні умови, освітлення, електромагнітні випромінювання, шум, вібрація, електричний струм, електростатичне поле, напруженість трудового процесу та інше.

\subsection{Організаційно-технічні заходи з поліпшення умов праці}

\subsubsection{Організування місця праці та безпечної роботи}

Описуючи організацію робочого місця і роботи варто зупинитись на кількох моментах, зокрема стосовно санітарно-гігієнічних та ергономічних вимог до параметрів робочого місця, розміщення обладнання, пристроїв та персонального комп’ютера на ньому, психофізіологічних особливостей праці.

Наукова організація робочого місця передбачала створення всіх необхідних умов для високопродуктивної і високоякісної праці за можливо менших фізичних зусиль і прагненні мінімізувати нервову напруженість. Вона виявилась у оснащеності робочого місця відповідним основним і допоміжним устаткуванням, технологічною і організаційною оснасткою; раціональним плануванням; створенням безпечних і здорових умов праці.

Під час роботи з персональним комп’ютером було дотримано наступні вимоги.
Площа приміщення, в якому розташовувались персональний комп’ютер та технічні засоби, визначались за наступними параметрами: площа — не менше 6,0 м$^2$; об’єм — не менше 20,0 м$^3$. Робоче місце розташовувалось на відстані 1,5 м від стіни з вікном.

Конструкція робочого місця забезпечила підтримання оптимальної робочої пози з такими ергономічними характеристиками:  ступні ніг – на підставці для ніг; стегна – в горизонтальній площині; передпліччя –  вертикально;  лікті – під кутом 70–90° до вертикальної площини; зап’ястя зігнуті під кутом не більше 20° відносно горизонтальної площини; нахил голови – 15-20° відносно вертикальної площини.

ПК і його периферійні пристрої (принтер, сканер) розміщувались на основному робочому столі, з лівого боку. Висота робочої поверхні столу для ПК складала 700 мм. Розміри столу: висота 750 мм, ширина 900 мм, глибина 900 мм.

Робочий стіл для ПК мав простір для ніг висотою 700 мм, шириною 600 мм, глибиною на рівні колін 500 мм, на рівні витягнутої ноги – 750 мм. Робоче сидіння ПК було оснащено такими елементами: сидіння, спинка. Воно було підйомно-поворотним, таким, що регулюється за висотою, кутом нахилу сидіння та спинки, за відстанню спинки до переднього краю сидіння, висотою підлокітників.

Монітор та клавіатура розташовувались на оптимальній відстані від очей. Діагональ монітору склала  43 см (17"), відповідно відстань до очей складала 750 мм. Клавіатура розміщувалась на поверхні на відстані 100–300 мм від краю. Кут нахилу клавіатури в межах 5–15 градусів. Розміщення принтера не перешкоджало добрій видимості монітору, зручність ручного керування пристроєм введення-виведення інформації в зоні досяжності моторного поля: по висоті 1200 мм, по глибині 450 мм.

З метою запобігання перевантаження організму як в цілому, так і окремих його функціональних систем, передусім зорового та рухового аналізаторів, центральної нервової системи, загальний час щоденної роботи з ПК було обмежено 50 \% часу робочого дня.

З урахуванням характеру трудової діяльності, напруженості та важкості праці з використанням ПК під час основної роботи за восьмигодинної робочої зміни було встановлено додаткові регламентовані перерви тривалістю 10 хв через кожну годину роботи.

Санітарно-гігієнічні вимоги до умов праці під час виконання роботи було дотримано у повному обсязі. Параметри мікроклімату у приміщенні забезпечили комфортне самопочуття організму.

Світло розташовувалось з лівого боку та забезпечило коефіцієнт природної освітленості не нижче 1,5 \%. Освітленість за штучного освітлення в площині робочої поверхні становило 300–500 Лк. Щоб уникнути світлових відблисків від екрану та клавіатури використовувалось комп’ютерне обладнання з матовою поверхнею. Для захисту очей  від прямого сонячного світла чи джерел штучного освітлення застосовувались жалюзі на вікнах.

Вимоги до рівнів шуму та вібрації було дотримані. "Гігієнічна регламентація шумів ґрунтується на критерії збереження здоров'я та працездатності людини", а тому рівень шуму, що супроводжував роботу, коливався у межах 40–50 дБА. Вимоги до рівня електромагнітних випромінювань, електростатичних та магнітних полів також були дотримані, оскільки не перевищували допустимих норм.

Облаштовуючи приміщення для роботи з ПК, було передбачено припливно-витяжну вентиляцію, кондиціювання повітря, тому що "вентиляція є одним із найважливіших санітарно-гігієнічних заходів, що забезпечують нормалізацію повітряного середовища у приміщенні".

Заходи особистої гігієни на робочому місці передбачали щоденне вологе прибирання, утримання у чистоті робочого місця, наявність на робочому місці тільки необхідних для роботи засобів. На робочому місці було дотримано вимог правил внутрішнього розпорядку, зокрема, заборонено приймати їжу, пити, курити та інше.

\subsubsection{Санітарно-гігієнічні вимоги до умов праці}

Особиста гігієна була забезпечена через дотримання рядку положень:

1. Заходи безпеки під час експлуатації персонального комп’ютера та периферійних пристроїв: правильну організацію робочого місця та дотримання оптимальних режимів праці та відпочинку під час роботи з ПК; експлуатацію сертифікованого обладнання; дотримання заходів електробезпеки; забезпечення оптимальних параметрів мікроклімату; забезпечення раціонального освітлення робочого місця; зниження рівня шуму та вібрації.

2. Заходи безпеки під час експлуатації інших електричних приладів: стеження за справним станом електромережі, розподільних щитків, вимикачів, штепсельних розеток, лампових патронів, а також мережевих кабелів живлення, за допомогою яких електроприлади під’єднують до електромережі; постійним  стеженням за справністю ізоляції електромережі та мережевих кабелів, не допускаючи їхньої експлуатації з пошкодженою ізоляцією.


\subsubsection{Заходи щодо безпеки під час виконання дипломної (кваліфікаційної) роботи та професійної діяльності}

Щодо аварійних ситуацій на робочому місці техногенного характеру та загроз природного характеру, що можуть перерости у надзвичайні ситуації, то таких вдалось уникнути за рахунок проекту протипожежних і противибухових заходів (див рис.)

\subsection{Безпека в надзвичайних ситуаціях}

\subsubsection{Протипожежні та противибухові заходи}

Оскільки специфіка написання кваліфікаційної роботи не передбачає використання пожежовибухонебезпечних речовин і матеріалів, то і аварійних ситуацій через їх використання вдалось уникнути.

Водночас і приміщення, в межах якого відбувалось виконання кваліфікаційної роботи, не підпадало під категорію пожежонебезпечності – оскільки процес виконання дослідження не вимагав використання швидко займистих та вибухонебезпечних речовин.

Пожеж вдалось уникнути за рахунок: дотримання пожежних норм і правил; виконання правил встановлення та експлуатації систем енергопостачання, опалення, вентиляції, правил експлуатації електричного та газового обладнання.


\subsubsection{Організування евакуації працівників}

Також було розроблено схему евакуації з приміщення, у якому виконувалась робота.

Було розроблено схему, що передбачає наявність евакуаційних шляхів та виходів зокрема це: дверні отвори, що ведуть назовні.
Шляхи та виходи відповідали таким вимогам: вони утримувались вільними, не захаращувалися та у разі потреби могли забезпечити евакуацію; кількість та розміри евакуаційних виходів, їхні конструктивні рішення, умови освітленості відповідали протипожежним вимогам будівельних норм; двері на шляхах евакуації відчинялись в напрямку виходу з будівлі і замикалися лише на внутрішні запори, які легко можна було відімкнути.

\subsection{Підсумки}
Отже, підбиваючи підсумки означеним положенням техніки безпеки при виконанні кваліфікаційної роботи варто зазначити, що специфіка організації простору її виконання повністю відповідали визначеним нормам. Умови санітарно-гігієнічних та ергономічних нормативів повністю дотримані. Все вищевідзначене дає можливість констатувати, що робота виконана у відповідності до всіх існуючих норм техніки безпеки та безпечної організації праці.



	% %!TEX root = ../Main.tex

\sect{Висновки}

В даній роботі з використанням чисельних схем методу скінчених елементів та апостеріорних оцінювачів похибки, побудовано ефективні чисельні процедури дослідження широкого класу двовимірних крайових задач.

Достовірність та точність як обчислювальних схемперевірено на задачах з відомими аналітичними розв’язками та порівняно з одним із існуючих оцінювачів.

Порівняно із існуючим оцінювачем із \cite{OstShynAee11} побудований в роботі наближає реальну похибку точніше.

За рахунок кращої точності оцінювача в одній із розглянутих задач було досягнено шуканої якості наближення, використовуючи вдвічі меншу кількість елементів.

На іншій задачі ситуація була протилежна - АОП \cite{OstShynAee11} використав вдвічі меншу кількість елементів. Проте, варто зауважити, що оцінювач побудований в роботі досягнув кращої реальної похибки.

Загалом можна говорити, що побудований оцінювач добре наближає реальну похибку апроксимаціїї МСЕ, навіть в сингулярно збурених задачах.



	\begin{frame}[allowframebreaks]
		\frametitle<presentation>{Список літератури}
		\nocite{*}
		\bibliography{bibliography}{}
		\bibliographystyle{amsplain}
	\end{frame}


\end{document}
