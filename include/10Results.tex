%!TEX root = ../Main.tex

\sect{Висновки}

В даній роботі з використанням чисельних схем методу скінчених елементів та апостеріорних оцінювачів похибки, побудовано ефективні чисельні процедури дослідження широкого класу двовимірних крайових задач.

Достовірність та точність обчислювальних схем перевірено на задачах з відомими аналітичними розв’язками та порівняно з одним із існуючих оцінювачів.

Порівняно із існуючим оцінювачем \cite{OstShynAee11}, АОП побудований в роботі наближає реальну похибку точніше.

За рахунок кращої точності оцінювача в одній із розглянутих задач було досягнено шуканої якості наближення, використовуючи вдвічі меншу кількість елементів.

На іншій задачі ситуація була протилежна - оцінювач \cite{OstShynAee11} використав вдвічі меншу кількість елементів. Проте, варто зауважити, що АОП побудований в роботі досягнув кращої реальної похибки.

Загалом можна говорити, що побудований оцінювач добре наближає реальну похибку апроксимаціїї МСЕ, навіть в сингулярно збурених задачах.

