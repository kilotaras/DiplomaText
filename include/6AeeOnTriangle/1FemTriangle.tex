%!TEX root = ../../Main.tex
\subsection{Лінійні апроксимації МСЕ на трикутнику}

Допустимо, що область $\Omega \subset \R^2$ точок $x = (x_1, x_2)$ розділена на скінченні трикутні елементи $K = A_iA_jA_k$,
	так що отримана тріангуляція $T_h = \lbrace K \rbrace$ задовільняє
\eqref{eq:split_properties}.
Позначимо $A_h = \lbrace A_i \rbrace_{i=1}^n$ і $E_h = \lbrace S_{ij} \rbrace$ множини вершин $A_i := (x_1^i, x_2^i)$ та ребер $S_{ij} = (A_i, A_j)$ відповідно.

Допустимо, що шукана апроксимація
\begin{equation}
	u_h \in V_h := \left\lbrace v \in C(\Omega) : v|_K \in P_1(k) \quad \forall K \in T_h \right\rbrace
\end{equation}
%
розв'язку задачі \eqref{eq:general_variational} має вигляд
%
\begin{equation}\label{eq:appr_kind}
	u_h(x)|_K = \sum \limits_{m=i,j,k} u_m L_m (x) \qquad \forall x=(x_1, x_2) \in K \quad \forall K \in T_h
\end{equation}
%
де $P_1(K)$ - множина поліномів першого ступеня на трикутнику $K$, а $L_m = L_m(x)$ -
	барицентричні координати трикутника $K = A_i A_j A_k$ (див. \autoref{fig:barycentric_coordinates})
%
\begin{equation}\label{eq:barycentric_coord}
	\begin{cases}
		L_i(x) = \frac{a_i + b_i x_1 + c_i x_2}{2 |K|} \\
			\left.
			\begin{split}
				&a_i := x_1^k x_2^j - x_1^j x_2^k \\
				&b_i := x_2^j -x_2^k \\
				&c_i := -x_1^j+x_1^k
			\end{split}
			\right|
			i \to j \to k \to i
	\end{cases}
\end{equation}
%
\begin{figure}[H]
	\centering
    \includegraphics[scale=0.5]{images/barycentric}
    \caption{Графік однієї із барицентричних координат}
    \label{fig:barycentric_coordinates}
\end{figure}
