%!TEX root = ../../Main.tex
\begin{frame}[allowframebreaks]
	\frametitle<presentation>{Лінійні апроксимації МСЕ на трикутнику}

	$\Omega \subset \R^2$ поділена на трикутники $K = A_iA_jA_k$

	Позначимо:

	\begin{itemize}
		\item  $A_h = \lbrace A_i \rbrace_{i=1}^n$ - множина вершин $A_i := (x_1^i, x_2^i)$
		\item $E_h = \lbrace S_{ij} \rbrace$ - множина ребер $S_{ij} = (A_i, A_j)$
	\end{itemize}

	\framebreak

		Будемо шукати апроксимацію
		\begin{equation}
			u_h \in V_h := \left\lbrace v \in C(\Omega) : v|_K \in P_1(K), \quad \forall K \in T_h \right\rbrace
		\end{equation}
		%
		розв'язку варіаційної задачі у вигляді
		%
		\begin{equation}\label{eq:appr_kind}
			u_h(x)|_K = \sum \limits_{m=i,j,k} u_m L_m (x), \qquad \forall x=(x_1, x_2) \in K, \quad \forall K \in T_h,
		\end{equation}

		$P_1(K)$ - множина поліномів першого ступеня на трикутнику $K$

	\framebreak

		$L_m = L_m(x)$ - барицентричні координати трикутника $K = A_i A_j A_k$

		\begin{equation}\label{eq:barycentric_coord}
			\begin{cases}
				L_i(x) = \frac{a_i + b_i x_1 + c_i x_2}{2 |K|} \\
					\left.
					\begin{split}
						&a_i := x_1^k x_2^j - x_1^j x_2^k \\
						&b_i := x_2^j -x_2^k \\
						&c_i := -x_1^j+x_1^k
					\end{split}
					\right|
					i \to j \to k \to i
			\end{cases}
		\end{equation}

	\end{frame}

\begin{frame}{Графік однієї із барицентричних координат}
	\begin{figure}[H]
		\centering
	    \includegraphics[scale=0.5]{images/barycentric}
	    \label{fig:barycentric_coordinates}
	\end{figure}
\end{frame}
