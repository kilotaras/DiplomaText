%!TEX root = ../../Main.tex

\subsection{Оцінювач на трикутнику}

Щоб ефективно розв'язати задачу
\eqref{eq:AEE_formulation} побудуємо базис $\lbrace \phi_{ij} \rbrace_{(i,j) \in S_m}$, так що

\begin{equation}\label{eq:basis_properties}
\begin{cases}
	\phi_{ij} \in V \setminus V_h \\
	\text{supp } \phi_{ij} = K \cup K^\prime, \quad K = A_iA_jA_k, \quad K^\prime = A_jA_iA_{k^\prime}
\end{cases}
\end{equation}

Таким чином структура оцінювача похибки $e_h$ наступна

\begin{equation}
	e_h(x) := \sum \limits_{(i,j) \in S_m} \lambda_{ij} \phi_{ij}(x) \qquad \forall x \in \Omega
\end{equation}

з невідомими коефіцієнтами $\lambda_{ij}$.

Функції $\phi_{ij}$ побудуємо наступним чином

\begin{equation}
	\phi_{ij}(x) =
	\begin{cases}
		L_i(x)L_j(x), &\quad \forall x \in K\\
		L_i^\prime(x)L_j^\prime(x), &\quad \forall x \in K^\prime\\
		0 &\quad \forall x \in \Omega \setminus (K \cup K^\prime)
	\end{cases}
\end{equation}

\TODO{IMAGE}

Таким чином отримали систему рівнянь

\begin{equation}\label{eq:AEE_system}
	\sum \limits_{q = (i,j) \in S_m} a_\Omega(\phi_q, \phi_t) \lambda_q = \langle \rho(u_h), \phi_t \rangle, \quad t = (i,j) \in S_m
\end{equation}

Для пришвидшення обчислень допустимо, що

\begin{equation}
	a_\Omega(\phi_q, \phi_t) \neq 0 \Rightarrow q = t
\end{equation}

Таким чином ми від системи \eqref{eq:AEE_system} переходимо до набору рівнянь

\begin{equation}
	\begin{split}
		a_\Omega(\phi_q, \phi_q) \lambda_q = \langle \rho(u_h), \phi_q \rangle\\
		\lambda_q = \frac{\langle \rho(u_h), \phi_q \rangle}{a_\Omega(\phi_q, \phi_q)}
			= \frac{\langle l_\Omega, \phi_q\rangle - a_\Omega(u_h, \phi_q)}{a_\Omega(\phi_q, \phi_q)}
	\end{split}
	\quad q = (i,j) \in S_m
\end{equation}

Провіши необхідні обчислення отримуємо оцінювач похибки у вигляді
\begin{equation}\label{eq:AEE_final}
	e_h(x) := \sum \limits_{(i,j) \in S_m} \lambda_{ij} \phi_{ij}(x) \qquad \forall x \in \Omega
\end{equation}
