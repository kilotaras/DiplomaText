%!TEX root = ../../Main.tex

\section{АОП на трикутних скінченних елементах}
\frame{\sectionpage}

\begin{frame}{Структура оцінювача}

	Будуємо базис $\lbrace \phi_{ij} \rbrace_{S_{ij} \in E_h}$, так що
%
	\begin{equation}\label{eq:basis_properties}
		\begin{cases}
			\phi_{ij} \in V \setminus V_h, \\
			\text{supp } \phi_{ij} = K \cup K^\prime, \quad K = A_iA_jA_k, \quad K^\prime = A_jA_iA_{k^\prime}.
		\end{cases}
	\end{equation}

	Структура оцінювача:
	$e_h(x) := \sum \limits_{S_{ij} \in E_h} \lambda_{ij} \phi_{ij}(x), \qquad \forall x \in \Omega$

	Функції $\phi_{ij}$ побудуємо наступним чином (див. \autoref{fig:AEE_basis}):
	%
	\begin{equation}
		\phi_{ij}(x) =
		\begin{cases}
			L_i(x)L_j(x), &\quad \forall x \in K ,\\
			L_i^\prime(x)L_j^\prime(x), &\quad \forall x \in K^\prime ,\\
			0 &\quad \forall x \in \Omega \setminus (K \cup K^\prime).
		\end{cases}
	\end{equation}

\end{frame}

\begin{frame}{Приклад $\phi_{ij}$ на сусідніх скінченних елементах.}

		\begin{figure}[H]
			\centering
		    \includegraphics[height=0.75\textheight]{images/basis}
		    \label{fig:AEE_basis}
		\end{figure}

\end{frame}

\newcommand{\error}[1]{\lambda_{#1} \phi_{#1}(x)}

\begin{frame}{Рівняння для знаходження АОП}

	\begin{equation}
		\begin{split}
			\lambda_{ij} = \frac{\langle \rho(u_h), \phi_{ij} \rangle}{a_\Omega(\phi_{ij}, \phi_{ij})}
				= \frac{\langle l_\Omega, \phi_{ij}\rangle - a_\Omega(u_h, \phi_{ij})}{a_\Omega(\phi_{ij}, \phi_{ij})}
		\end{split}
		\quad S_{ij} \in E_h.
	\end{equation}

	Отримали АОП у вигляді

	\begin{equation}\label{eq:AEE_final}
		e_h(x)\bigg|_K = \error{ij}+\error{jk}+\error{ki}, \qquad K = A_iA_jA_k.
	\end{equation}

\end{frame}
