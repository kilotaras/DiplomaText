%!TEX root = ../../Main.tex

\subsection{Оцінювач на трикутнику}

Щоб ефективно розв'язати задачу
\eqref{eq:AEE_formulation} для отриманої апроксимації \eqref{eq:appr_kind}, побудуємо базис $\lbrace \phi_{ij} \rbrace_{S_{ij} \in E_h}$, так що
%
\begin{equation}\label{eq:basis_properties}
\begin{cases}
	\phi_{ij} \in V \setminus V_h, \\
	\text{supp } \phi_{ij} = K \cup K^\prime, \quad K = A_iA_jA_k, \quad K^\prime = A_jA_iA_{k^\prime}.
\end{cases}
\end{equation}

Таким чином, структура оцінювача похибки $e_h$ має вигляд:
%
\begin{equation}
	e_h(x) := \sum \limits_{S_{ij} \in E_h} \lambda_{ij} \phi_{ij}(x), \qquad \forall x \in \Omega
\end{equation}
%
з невідомими коефіцієнтами $\lambda_{ij}$.

Функції $\phi_{ij}$ побудуємо наступним чином (див. \autoref{fig:AEE_basis}):
%
\begin{equation}
	\phi_{ij}(x) =
	\begin{cases}
		L_i(x)L_j(x), &\quad \forall x \in K ,\\
		L_i^\prime(x)L_j^\prime(x), &\quad \forall x \in K^\prime ,\\
		0 &\quad \forall x \in \Omega \setminus (K \cup K^\prime).
	\end{cases}
\end{equation}
%
\begin{figure}[H]
	\centering
    \includegraphics[scale=0.7]{images/basis}
    \caption{Приклад $\phi_{ij}$ на сусідніх скінченних елементах.}
    \label{fig:AEE_basis}
\end{figure}

Таким чином, отримали систему рівнянь
%
\begin{equation}\label{eq:AEE_system}
	\sum \limits_{S_{ij} \in E_h} a_\Omega(\phi_{ij}, \phi_{rs}) \lambda_{ij} = \langle \rho(u_h), \phi_{rs} \rangle, \quad S_{rs} \in E_h.
\end{equation}

Для пришвидшення обчислень візьмемо, що
%
\begin{equation}
	a_\Omega(\phi_{ij}, \phi_{rs}) = 0, \text{ якщо } S_{ij} \ne S_{rs}.
\end{equation}

Таким чином, від системи \eqref{eq:AEE_system} переходимо до набору рівнянь
%
\begin{equation}
	\begin{split}
		a_\Omega(\phi_{ij}, \phi_{ij}) \lambda_{ij} = \langle \rho(u_h), \phi_{ij} \rangle ,\\
		\lambda_{ij} = \frac{\langle \rho(u_h), \phi_{ij} \rangle}{a_\Omega(\phi_{ij}, \phi_{ij})}
			= \frac{\langle l_\Omega, \phi_{ij}\rangle - a_\Omega(u_h, \phi_{ij})}{a_\Omega(\phi_{ij}, \phi_{ij})}
	\end{split}
	\quad S_{ij} \in E_h.
\end{equation}

Провівши необхідні обчислення, отримуємо оцінювач похибки у вигляді
\begin{equation}\label{eq:AEE_final}
	e_h(x) := \sum \limits_{S_ij \in E_h} \lambda_{ij} \phi_{ij}(x), \qquad \forall x \in \Omega.
\end{equation}
