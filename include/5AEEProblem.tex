%!TEX root = ../Main.tex

\begin{frame}[allowframebreaks]
	\frametitle<presentation>{Апостеріорна оцінка похибки}
	Для розв'язку задачі про похибку
		\eqref{eq:AE_problem} використаємо процес Гальоркіна, вибравши скінченновимірний простір $E_h \subset E$:
		%
		\begin{equation}\label{eq:AEE_formulation}
			\begin{cases}
				\mbox{для заданого } T_h=\{K\} \text{, і відповідного наближення } u_h \in V_h \\
				\text{та підпростору } E_h \subset E:=V \setminus V_h, \quad \text{dim} E_h = M(h) < +\infty \\
				\text{знайти оцінку } e_h \in E_h \text{, таку що} \\
				a_\Omega(e,v) = \langle\rho(u_h), v\rangle, \qquad \forall v \in E_h.
			\end{cases}
		\end{equation}

		\framebreak

			Задача коректно поставлена і її розв'язок $e_h \in E_h$ володіє наступними властивостями \footfullcite{OstShynAee11}:
			%
			\begin{equation}
				\begin{split}
					\norm{e_h}_V^2 = a_\Omega(e_h, e_h) = \langle\rho(u_h), e_h\rangle \le \\
					\le \norm{\rho(u_h)}_* \norm{e_h}_V \le \norm{u-u_h}_V \norm{e_h}_V,
				\end{split}
			\end{equation}
			%

			%
			\begin{empheq}[innerbox=\fbox]{equation}
				\norm{e_h}_V \le \norm{u-u_h}_V, \quad \forall h > 0.
			\end{empheq}

\end{frame}
