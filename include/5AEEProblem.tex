%!TEX root = ../Main.tex

\section{Апостеріорна оцінка похибки}

Для розв'язку задачі
\eqref{eq:AE_problem} використаємо процес Гальоркіна, вибравши певний скінченновимірний простір $E_h \subset E$:
%
\begin{equation}\label{eq:AEE_formulation}
	\begin{cases}
		\mbox{для заданого } T_h=\{K\} \text{, і відповідного наближення } u_h \in V_h \\
		\text{та підпростору } E_h \subset E:=V \setminus V_h, \quad \text{dim} E_h = M(h) < +\infty \\
		\text{знайти оцінку } e_h \in E_h \text{, таку що} \\
		a_\Omega(e,v) = \langle\rho(u_h), v\rangle, \qquad \forall v \in E_h.
	\end{cases}
\end{equation}

Задача \eqref{eq:AEE_formulation} коректно поставлена і її розв'язок $e_h \in E_h$ володіє наступними властивостями \cite{OstShynAee11}:
%
\begin{equation}
	\begin{split}
		\norm{e_h}_V^2 = a_\Omega(e_h, e_h) = \langle\rho(u_h), e_h\rangle \le \\
		\le \norm{\rho(u_h)}_* \norm{e_h}_V \le \norm{u-u_h}_V \norm{e_h}_V,
	\end{split}
\end{equation}
%
тобто
%
\begin{equation}
	\norm{e_h}_V \le \norm{u-u_h}_V, \quad \forall h > 0.
\end{equation}

Обираючи базис $\lbrace \phi_i(x)\rbrace_{i=1}^M \subset E_h$, конкретизуємо задачу
\eqref{eq:AE_problem} до алгебраїчного представлення
%
\begin{equation}
	\begin{cases}
		\text{для заданих } T_h = \lbrace K \rbrace \text{ і базису } \lbrace \phi_i(x)\rbrace_{i=1}^N \text{ з простору } E_h \subset E\\
		\text{знайти коефіцієнти } \lambda_m \in \R \text{ лінійної комбінації } \\
			\qquad \qquad \qquad e_h(x) := \sum\limits_{m=1}^M \lambda_m \phi_m (x) \in E_h, \text{ що} \\
		\sum\limits_{m=1}^M a_\Omega(\phi_m, \phi_i)\lambda_m = \langle \rho(u_h), \phi_i \rangle, \qquad i = 1,\dots,M(h).
	\end{cases}
\end{equation}

Очевидно, що якість отриманого оцінювача $e_h \in E_h$ напряму залежить від повноти простору $E_h \subset E$ та ефективності обрахунків.
Ефективність досягається завдяки схемі скінченних елементів, що дає змогу вибрати базис, близький до ортогонального.

Далі ми будуємо чисельну схему, що задовольняє ці критерії у 2D випадку ($d=2$).
