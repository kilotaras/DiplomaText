%!TEX root = ../Main.tex

\sect{Вступ}

Метод скінченних елементів (МСЕ) -- це числова техніка знаходження розв'язків інтегральних та часткових диференціальних рівнянь.

Аналогічно до того, як багато маленьких відрізків можуть наблизити коло, метод скінченних елементів пробує наблизити складний розв'язок набором простих функцій на окремих ділянках простору, які називаються скінченними елементами.

При розв'язанні часткових диференціальних рівнянь (ЧДР), головною метою є створення рівності, що апроксимує досліджувану рівність, і є числово стабільною, тобто помилки у вхідних даних і проміжних обчисленнях не акумулюються і не спричиняють беззмістовних результатів. Для реалізації цього є багато способів, кожен зі своми плюсами і мінусами.

Метод скінченних елементів є добрим вибором при розв'язуванні ЧДР, які описують складні середовища (такі як машини, чи нафтогони), при змінності цих середовищ,чи коли бажана точність змінюється у різних ділянках середовища.

Наприклад, при моделювання фронтального розбиття машини, є можливість збільшити точність моделювання у важливіших зонах, таких, як передня частина машини, і зменшити її при обрахунку того, що відбудеться із задньою частиною машини (тим самим зменшивши ресурсоємність моделювання).

Іншим прикладом може служити моделювання погоди на Землі, при якому важливішою є погода над сушею, ніж над безкраїми морськими просторами.

Апостеріорні оцінювачі для методу скінченних елементів є важливою частиною сучасної інформатики (див. \cite{verfurth1996review, verfurth1994posteriori,eriksson1995introduction,ainsworth2011posteriori}). За останні десятиліття на базі початкової ідеї Бабушки і Рейнбольдта \cite{babuvska1978posteriori} було побудовано цілу сім'ю різних апостеріорних оцінювачів похибки (АОП). З їх допомогою можна якісно оцінити похибку отриманої апроксимації МСЕ і створити базу для локального покращення сітки, що дає змогу знаходити розв'язки з заданою точністю і мінімальними обчислювальними затратами \cite{babuska2011finite}

В даній роботі ми будуємо апостеріорний оцінювач похибки для двовимірної задач та адаптивну схему для МСЕ на його основі. Для побудованої схеми проводимо числовий аналіз якості для задачі з відомим точним розв'язком.
Отримані результати порівнюються з результатами роботи оцінювача запропонованого в \cite{OstShynAee11}


