%!TEX root = ../Main.tex

\sect{Вступ}

Апостеріорні оцінювачі для методу скінченних елементів(МСЕ) є важливою частиною сучасної інформатики (див. \cite{verfurth1996review, verfurth1994posteriori,eriksson1995introduction,ainsworth2011posteriori}). За останні десятиліття на базі початкової ідеї Бабушки і Рейнбольдта \cite{babuvska1978posteriori} було побудовано цілу сім'ю різних апостеріорних оцінювачів похибки (АОП). З їх допомогою можна якісно оцінити похибку отриманої апроксимації МСЕ і створити базу для локального покращення сітки, що дає змогу знаходити розв'язки з заданою точністю і мінімальними обчислювальними затратами \cite{babuska2011finite}

В даній роботі ми будуємо апостеріорний оцінювач похибки для двовимірної задач та адаптивну схему для МСЕ на його основі. Для побудованої схеми проводимо числовий аналіз якості для задачі з відомим точним розв'язком.
Отримані результати порівнюються з результатами роботи оцінювача запропонованого в \cite{OstShynAee11}


