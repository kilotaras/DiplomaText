%!TEX root = ../Main.tex

\section {Дискретизована задача}

Допустимо, що $\Omega \subset \R$ поділена в скінченні елементи $K$ таким чином, що поділ
$
	T_h=\lbrace K \rbrace, \quad
	h := \max_{K \in T_h}h_K, \quad
	h_k := \mbox{diam} K
$
має наступні властивості:

\begin{equation}
\begin{split}
	& \Omega = \bigcup_{K \in T_h} K \\
	& K \bigcap K^\prime = \emptyset \quad \forall K, K^\prime \in T_h : K \neq K^\prime \\
	& \overline K \bigcap \overline {K^\prime} =
	\begin{cases}
		S := \mbox{\{спільне ребро\}} \\
		A := \mbox{\{спільна вершина\}}  \\
		\emptyset \\
	\end{cases}
\end{split}
\end{equation}

Після побудови певним чином скінченновимірного апроксимуючого простору $V_h \subset V$, $\mbox{dim} V_h = N(h) = N < + \infty$
на тріангуляції $T_h = \lbrace K \rbrace$ ми можемо замінити задачу \ref{eq:general_variational} наступною задачою

\begin{equation}\label{eq:general_discrete}
	\begin{cases}
		\mbox{для заданих } T_h = \{K\}, V_h  \subset V \\
		\mbox{знайти } u_h \in V_h \mbox{, що} \\
		a_\Omega(u_h, v) = \langle l_\Omega, v \rangle \qquad \forall v \in V_h
	\end{cases}
\end{equation}
