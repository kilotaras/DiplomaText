%!TEX root = ../Main.tex

\section{Дискретизована задача та апроксимації МСЕ}
\frame{\sectionpage}
\begin{frame}{Поділ простору}
		 $\Omega \subset \R^d$ поділена на скінченні елементи $K$:
		 $
			T_h=\lbrace K \rbrace, \quad
			h := \max_{K \in T_h}h_K, \quad
			h_k := \mbox{diam} K
		$
		\begin{equation}\label{eq:split_properties}
			\begin{split}
				& \Omega = \bigcup_{K \in T_h} K, \\
				& K \bigcap K^\prime = \emptyset \quad \forall K, K^\prime \in T_h : K \neq K^\prime, \\
				& \overline K \bigcap \overline {K^\prime} =
				\begin{cases}
					S := \mbox{\{спільне ребро\}}, \\
					A := \mbox{\{спільна вершина\}},  \\
					\emptyset. \\
				\end{cases}
			\end{split}
		\end{equation}
\end{frame}

\begin{frame}{Дискретизована задача}
		$V_h \subset V$, $\mbox{dim} V_h = N(h) = N < + \infty$ - простір апроксимацій

		Заміняємо задачу \eqref{eq:general_variational} на наступну

		\begin{equation}\label{eq:general_discrete}
			\begin{cases}
				\mbox{для заданого } T_h = \{K\}, V_h  \subset V \\
				\mbox{знайти } u_h \in V_h \mbox{, таку що} \\
				a_\Omega(u_h, v) = \langle l_\Omega, v \rangle, \qquad \forall v \in V_h.
			\end{cases}
		\end{equation}

\end{frame}

\begin{frame}{Система рівнянь}
		Обравши базис $\lbrace \phi_i(x)\rbrace_{i=1}^N \subset V_h$,конкретизуємо \eqref{eq:general_discrete}:
		%
		\begin{equation}\label{eq:concrete_discrete}
		\begin{cases}
			\text{для заданих } T_h = \lbrace K \rbrace \text{ і базису } \lbrace \phi_i(x)\rbrace_{i=1}^N \text{ з простору } V_h \subset V \\
			\text{знайти коефіцієнти } q_m \in \R \text{ апроксимації } \\
				\qquad u_h(x) := \sum\limits_{m=1}^N q_m \phi_m (x) \in V_h, \text{ такі що} \\
			\sum\limits_{m=1}^N a_\Omega(\phi_m, \phi_i)q_m = \langle l_\Omega, \phi_i \rangle, \qquad i = 1,\dots,N.
		\end{cases}
		\end{equation}

		Матриця системи рівнянь буде додатньо визначеною \footfullcite{OstShynAee11}

\end{frame}
