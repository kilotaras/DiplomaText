%!TEX root = ../Main.tex
\clearpage
\section{Формулювання задачі}

Нехай $\Omega$ - область в Евклідовому просторі $\R^d$ з неперервним краєм $\Gamma \equiv \delta \Omega$, яка задовольняє умову Ліпшица.

Розглянемо крайову задачу \eqref{eq:general_boundary} та її варіаційне формулювання \eqref{eq:general_variational}:
%
\begin{equation}\label{eq:general_boundary}
	\begin{cases}
			- \nabla (\mu \nabla u) + \beta \cdot \nabla u + &\sigma u = f,  \\
			&u|_\Gamma = 0.
	\end{cases}
\end{equation}
%
\begin{equation}\label{eq:general_variational}
	\begin{cases}
		\mbox{Для } V := H_0^1 \left( \Omega \right) =
		\left\lbrace
			v \in H^1 \Omega : v|_\Gamma = 0
		\right\rbrace, \\

		\quad a_\Omega(w,v) := \displaystyle\int_\Omega
		\left[
			\mu \nabla w \nabla v +\beta v \nabla w + \sigma wv
		\right] dx, \\

		\qquad \left\langle l_\Omega, v \right\rangle := \displaystyle\int_\Omega fvdx \quad \forall v,w \in V \\

		\mbox{знайти }u \in V \mbox{, таку що} \\

		\qquad a_\Omega(w,v) = \left\langle l_\Omega, v \right\rangle \quad \forall v \in V.

	\end{cases}
\end{equation}

Тут $\nabla u := \left\lbrace \frac{\partial u}{\partial x_i} \right\rbrace_{i=1}^d$,
	а також $\mu = \left\lbrace \mu_{ij}(x) \right\rbrace_{i,j=1}^d$,
	$\beta = \left\lbrace \beta_i(x) \right\rbrace_{i=1}^d$,
	$\sigma = \sigma(x)$,
	$f = f(x)$ - задані функції, такі що
%
%
\begin{equation}
	\begin{cases}
		\mu_{ij}(x) = \mu_{ji}(x) ,\\
			\sum \limits_{i,j=1}^d \mu_{ij}(x) \xi_i \xi_j
				\ge
			\mu_0 \sum \limits_{i=1}^d \xi_i^2, \quad
			\mu_0 =\text{const}, \quad \forall \xi_i \in \R ,\\
		\nabla \cdot \beta(x) := \mathlarger{\sum \limits_{i=1}^d} \frac{\partial \beta_i(x)}{\partial x_i} = 0
			\text{ майже скрізь в } \Omega.
	\end{cases}
\end{equation}
%
\begin{equation}\label{eq:coef_spaces}
	\mu_{ij}, \beta_i, \sigma \in L^\infty(\Omega), \quad f \in L^2(\Omega)
\end{equation}
%
При таких умовах білінійна форма $a_\Omega (\cdot,\cdot): V \times V  \to \R$ є неперервною від вхідних даних, $V$-еліптичною і генерує енергетичну норму \cite{kozarevska2002}
%
\begin{equation}\label{eq:energy_norm}
	\norm{u}_V := \sqrt{a_\Omega(u,u)}
\end{equation}

\newcommand{\vonenorm}{\left|v\right|_{1,\Omega}}

Норма \eqref{eq:energy_norm} є еквівалентною до норми $\vonenorm = \sqrt{(\nabla v, \nabla v)}$ \cite{OstShynAee11} :
%
\newcommand{\infnorm}[1]{\norm{#1}_\infty}
%
\begin{equation}\label{eq:norm_equivalence}
	\mu_0 \vonenorm^2 \le \norm{v}^2_V \le
		\mu_0
			\left[
				\frac{\norm{\mu}_\infty}{\mu_0}+Pe(1+Sh)
			\right]
		\vonenorm^2 ,
\end{equation}
%
де
%
\begin{equation}
	Pe:= \frac{\infnorm{\beta} \text{ diam} \Omega}{\mu_0}, \qquad
	Sh:= \frac{\infnorm{\sigma} \text{ diam} \Omega}{\infnorm{\beta}}
\end{equation}
%
критерії Пекле і Струхаля. Тут
%
\begin{equation}
	\begin{split}
		\infnorm{\sigma} := ess \sup \limits_{x \in \Omega} |\sigma(x)|, \\
		\infnorm{\beta} := \sqrt{\sum \limits_{i=1}^d \infnorm{\beta_i}^2}, \\
		\infnorm{\mu} := \sqrt{\sum \limits_{i,j=1}^d \infnorm{\mu_{ij}}^2}.
	\end{split}
\end{equation}

\undef{\vonenorm}

Згідно з
\eqref{eq:coef_spaces} лінійний функціонал $l : V \to \R$ є неперервним.
Тому згідно теореми Лакса-Мільграма-Вишика задача
\eqref{eq:general_variational} коректно поставлена. Вона має єдиний розв'язок $u \in V$, що неперервно залежить від вхідних даних.
