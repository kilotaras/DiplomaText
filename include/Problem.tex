%!TEX root = ../Main.tex
\section{Формулювання задачі}

Нехай $\Omega$ - область в Евклідовому просторі $\mathbb{R}^d$ з неперервним краєм $\Gamma \equiv \delta \Omega$, що задовільняє умову Ліпшица.

Розглянемо крайову задачу

\begin{equation}\label{eq:general_boundary}
	\begin{cases}
			- \nabla [\mu \nabla u] + \beta \nabla u + &\sigma u = f  \\
			&u|_\Gamma = 0
	\end{cases}
\end{equation}

її варіаційне формулювання буде наступним:

\begin{equation}\label{eq:general_variational}
	\begin{cases}
		\mbox{для } V := H_0^1 \left( \Omega \right) = 
		\left\lbrace 
			v \in H^1 \Omega : v|_\Gamma = 0
		\right\rbrace \\
		
		\quad a_\Omega(w,v) := \displaystyle\int_\Omega 
		\left[
			\mu \nabla w \nabla v +v \beta \nabla w + \sigma wv
		\right] dx \\
		
		\qquad \left\langle l_\Omega, v \right\rangle := \displaystyle\int_\Omega fvdx \quad \forall v,w \in V \\
		
		\mbox{знайти }u \in V \mbox{ що} \\
		
		\qquad a_\Omega(w,v) = \left\langle l_\Omega, v \right\rangle \quad \forall v \in V
		
	\end{cases}
\end{equation}

Тут $\nabla u := \left\lbrace \frac{\partial u}{\partial x_i} \right\rbrace_{i=1}^d$, 
	а також $\mu = \left\lbrace \mu_{ij}(x) \right\rbrace_{i,j=1}^d$,
	$\beta = \left\lbrace \beta_i(x) \right\rbrace_{i=1}^d$,
	$\sigma = \sigma(x)$,
	$f = f(x)$
