%!TEX root = ../Main.tex

\section{Стратегія адаптації}

Отримавши оцінку похибки \eqref{eq:AEE_final}, можемо побудувати адаптивну стратегію для знаходження розв'язку із заданою точністю.
Розпочавши з деякої сітки $T_h =  \left\{ K \right\}$, ми можемо знайти ті скінченні елементи, на яких похибка найбільша.
Після цього можна ущільнити на знайдених елементах сітку і знову застосувати схему МСЕ.

Для початку природнім чином визначимо похибку на кожному скінченному елементі:
%
\newcommand{\error}[1]{\lambda_{#1} \phi_{#1}(x)}
\begin{equation}\label{eq:error_element}
	e_h(x)\bigg|_K = \error{ij}+\error{jk}+\error{ki}, \qquad K = A_iA_jA_k.
\end{equation}

Тоді, за показник якості розв'язку на трикутнику виберемо
%
\begin{equation}\label{eq:error_indicator}
	\text{ind}_K = \sqrt{N \frac
			{\norm{e_h}^2_{H_1(K)}}
			{\norm{u_h+e_h}^2_H_1(\Omega)}
	}*100\%.
\end{equation}

Якщо цей показник перевищує деяку, наперед задану, межу ділимо цей трикутник на менші.
Процес зупиняємо тоді, коли індикатор похибки на кожному скінченному елементі менший за задану межу.

\undef{\error}
