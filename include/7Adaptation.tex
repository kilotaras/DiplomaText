%!TEX root = ../Main.tex

\section{Стратегія адаптування тріангуляції}
\frame{\sectionpage}

\begin{frame}{Стратегія адаптації}

	\begin{itemize}
		\item Отримавши оцінку похибки будуємо адаптивну стратегію для знаходження розв'язку із заданою точністю
		\item Починаємо з якоїсь сітки
		\item Ущільнюємо сітку на елементах де велика похибка
	\end{itemize}

\end{frame}

\begin{frame}{Умова поділу}

	Маємо
	%
	\begin{equation}\label{eq:error_element}
		e_h(x)\bigg|_K = \error{ij}+\error{jk}+\error{ki}, \qquad K = A_iA_jA_k.
	\end{equation}

	Показник якості розв'язку на трикутнику:
	%
	\begin{equation}\label{eq:error_indicator}
		\text{ind}_K = \sqrt{N \frac
				{\norm{e_h}^2_{H(K)}}
				{\norm{u_h+e_h}^2_H(\Omega)}
		}*100\%.
	\end{equation}

	\begin{equation*}\label{eq:error_indicator}
		\text{ind}_K < \varepsilon \Rightarrow \text{ ділимо трикутник}
	\end{equation*}

\end{frame}

\undeff{\error}
