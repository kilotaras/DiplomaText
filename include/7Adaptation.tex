%!TEX root = ../Main.tex

\newcommand{\error}[1]{\lambda_{#1} \phi_{#1}(x)}

\begin{frame}[allowframebreaks]
	\frametitle<presentation>{Стратегія адаптації}

	\begin{itemize}
		\item Отримавши оцінку похибки будуємо адаптивну стратегію для знаходження розв'язку із заданою точністю
		\item Починаємо з якоїсь сітки
		\item Ущільнюємо сітку на елементах де велика похибка
	\end{itemize}

	\framebreak

	Визначимо похибку на кожному скінченному елементі:
	%
	\begin{equation}\label{eq:error_element}
		e_h(x)\bigg|_K = \error{ij}+\error{jk}+\error{ki}, \qquad K = A_iA_jA_k.
	\end{equation}

	Показник якості розв'язку на трикутнику:
	%
	\begin{equation}\label{eq:error_indicator}
		\text{ind}_K = \sqrt{N \frac
				{\norm{e_h}^2_{H(K)}}
				{\norm{u_h+e_h}^2_H(\Omega)}
		}*100\%.
	\end{equation}

	Показник перевищує деяку задану межу $\Rightarrow$ ділимо цей трикутник.
\end{frame}

\undeff{\error}
