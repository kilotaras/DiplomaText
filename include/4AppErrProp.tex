%!TEX root = ../Main.tex

\section{Задача оцінки похибки}

Враховуючи
\eqref{eq:general_variational} та дискретизовану версію
\eqref{eq:general_discrete} ми отримуємо наступну задачу:
%
\begin{equation}\label{eq:AE_problem}
	\begin{cases}
		\mbox{для заданого } T_h=\{K\} \text{, і відповідного наближення } u_h \in V_h \\
		\text{знайти похибку } e:=u-u_h \in E := V \setminus V_h \text{, таку що} \\
		a_\Omega(e,v) = \langle\rho(u_h), v\rangle \qquad \qquad \forall v \in E \text{, де}\\
		\langle\rho(w), v\rangle := \langle l_\Omega, v\rangle - a_\Omega(w, v) \quad \forall w,v \in V.
	\end{cases}
\end{equation}

Залишок $\langle\rho(u_h), v\rangle$ можна записати наступним чином
\begin{equation*}
	\langle\rho(u_h), v\rangle
		= \langle l_\Omega, v\rangle - a_\Omega(u_h, v)
		= a_\Omega(u,v)-a_\Omega(u_h, v) = a_\Omega(u-u_h, v)
\end{equation*}

Можна показати \cite{kvasnyca2002}, що
%
\begin{equation}\label{eq:residual_properties}
	\begin{split}
		\langle\rho(u_h), v\rangle = a_\Omega(u-u_h, v) = 0 \quad \forall v \in V_h \subset V \\
		\langle\rho(u_h), u-u_h\rangle = a_\Omega(u-u_h, u-u_h) = \norm{u-u_h}^2_V \\
		\norm{\rho(u_h)}_* = \sup \limits_{v \ne 0 \in V} \frac{|\langle\rho(u_h), v\rangle|}{\norm{v}_V} =
			\sup \limits_{v \ne 0 \in V} \frac{a_\Omega(u-u_h, v)}{\norm{v}_V} \le \norm{u-u_h}_V
	\end{split}
\end{equation}

З
\eqref{eq:residual_properties} слідує, що норма залишку еквівалента енергетичній нормі МСЕ
\begin{equation}
	\norm{\rho(u_h)}_* = \norm{u-u_h}_V
\end{equation}


