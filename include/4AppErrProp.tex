%!TEX root = ../Main.tex


\section{Задача про похибку апроксимацій МСЕ}
\frame{\sectionpage}
\begin{frame}[allowframebreaks]
	\frametitle<presentation>{Задача про похибку апроксимацій МСЕ}

		З варіаційної задачі \eqref{eq:general_variational} та її дискретизованої версії
		\eqref{eq:general_discrete} маємо
	%
		\begin{equation}\label{eq:AE_problem}
			\begin{cases}
				\mbox{для заданого } T_h=\{K\} \text{, і відповідного наближення } u_h \in V_h \\
				\text{знайти похибку } e:=u-u_h \in E := V \setminus V_h \text{, таку що} \\
				a_\Omega(e,v) = \langle\rho(u_h), v\rangle, \qquad \qquad \forall v \in E \text{, де}\\
				\langle\rho(w), v\rangle := \langle l_\Omega, v\rangle - a_\Omega(w, v), \quad \forall w,v \in V.
			\end{cases}
		\end{equation}

	\framebreak

		Залишок $\langle\rho(u_h), v\rangle$ можна записати наступним чином:
		\begin{equation*}
			\begin{split}
				\langle\rho(u_h), v\rangle
					= \langle l_\Omega, v\rangle - a_\Omega(u_h, v) = \\
					= a_\Omega(u,v)-a_\Omega(u_h, v) = a_\Omega(u-u_h, v).
			\end{split}
		\end{equation*}

		Можна показати \footfullcite{kvasnyca2002}, що
		%
		\begin{equation*}\label{eq:residual_properties}
			\begin{split}
				\langle\rho(u_h), v\rangle = a_\Omega(u-u_h, v) = 0, \quad \forall v \in V_h \subset V,\\
				\langle\rho(u_h), u-u_h\rangle = a_\Omega(u-u_h, u-u_h) = \norm{u-u_h}^2_V,\\
				\norm{\rho(u_h)}_* = \sup \limits_{v \ne 0 \in V} \frac{|\langle\rho(u_h), v\rangle|}{\norm{v}_V} = \\
					 = \sup \limits_{v \ne 0 \in V} \frac{a_\Omega(u-u_h, v)}{\norm{v}_V} \le \norm{u-u_h}_V .
			\end{split}
		\end{equation*}
\end{frame}
